\documentclass{article}
\usepackage{amsmath}
\usepackage{bm}
\usepackage{graphicx}
\usepackage{subcaption}
\input{macros}
\begin{document}

\section{Sheet G01 - Bézier Curves}
Robin Landsgesell, Jonas Marcic, Julian Stamm

\subsection{Assignment 4}

\subsubsection{a)}

\begin{align*}
B_0^2(t)&=\binom{2}{0}(1-t)^2&=(1-t)^2&=1-2t+1t^2\\
B_1^2(t)&=\binom{2}{1}t(1-t)&=2t(1-t)&=0+2t-2t^2\\
B_2^2(t)&=\binom{2}{2}t^2&=t^2&=0+0t+1t^2
\end{align*}
\begin{align*}
\implies
\mtx{B_0^2\\B_1^2\\B_2^2} \mat{P}
=
\mtx{1&t&t^2}
\mtx{
     1& 0& 0\\
    -2& 2& 0\\
    1&-2& 1
}
\mat{P}
\end{align*}

\subsubsection{b)}

\begin{align*}
    \left [
    \begin{array}{ccc}
        1 & (z\cdot t) & (z\cdot t)^2 \\
    \end{array}
    \right ] 
    =
    \left [
    \begin{array}{ccc}
        1 & t & t^2 \\
    \end{array}
    \right ] \cdot Z
    \implies
    Z =
    \left [
    \begin{array}{ccc}
        1 & 0 & 0 \\
        0 & z & 0 \\
        0 & 0 & z^2 \\
    \end{array}
    \right ]
\end{align*}

\subsubsection{c)}

\begin{align*}
    &\left [
    \begin{array}{ccc}
        1 & t & t^2 \\
    \end{array}
    \right ] \cdot Z \cdot M \cdot P
    =
    \left [
    \begin{array}{ccc}
        1 & t & t^2 \\
    \end{array}
    \right ] \cdot M \cdot Q \cdot P
    \\
    &\implies Z M = M Q\\
    &\implies Q = M^{-1} Z M\\
    &\implies Q = 
    \mtx{
     1& 0& 0\\
    1-z& z& 0\\
    (1-z)^2& 2z(1-z)& z^2
}
\end{align*}

\subsubsection{d)}

\begin{align*}
    &\left [
    \begin{array}{ccc}
        1 & (z + (1-z)\cdot t) & (z + (1-z)\cdot t)^2 \\
    \end{array}
    \right ] 
    =
    \left [
    \begin{array}{ccc}
        1 & t & t^2 \\
    \end{array}
    \right ] \cdot Z
    \implies
    Z =
    \left [
    \begin{array}{ccc}
        1 & z & z^2 \\
        0 & 1-z & 2z(1-z) \\
        0 & 0 & (1-z)^2 \\
    \end{array}
    \right ]\\
    &\implies Q = M^{-1} Z M\\
    &\implies Q = 
    \mtx{
    (1-z)^2& 2z(1-z)& z^2\\
    0& 1-z& z\\
    0& 0& 1
    }
\end{align*}

\subsection{Assignment 5}

\subsubsection{a)}
\begin{equation*}
    p(u) = (1-u)^2 P_0 + 2u(1-u) P_1 + u^2 P_2, \quad u \in [0,1]
\end{equation*}

\subsubsection{b)}
\begin{align*}
x(u) &= -1(1-u)^2 + 0\cdot 2u(1-u) + 2u^2 = -1 + 2u + u^2,\\[4pt]
y(u) &= 0(1-u)^2 + 1\cdot 2u(1-u) + 0\cdot u^2 = 2u - 2u^2.\\
\implies p(u) &= \big(-1 + 2u + u^2,\; 2u - 2u^2\big)
\end{align*}

\subsubsection{c)}
\begin{figure}[htbp]
\centering
\begin{subfigure}[b]{0.32\textwidth}
    \includegraphics[width=\linewidth]{imgs/IMG_0305.jpg}
    \caption{$B_0(u)=(1-u)^2$}
\end{subfigure}
\hfill
\begin{subfigure}[b]{0.32\textwidth}
    \includegraphics[width=\linewidth]{imgs/IMG_0306.jpg}
    \caption{$B_1(u)=2u(1-u)$}
\end{subfigure}
\hfill
\begin{subfigure}[b]{0.32\textwidth}
    \includegraphics[width=\linewidth]{imgs/IMG_0307.jpg}
    \caption{$B_2(u)=u^2$}
\end{subfigure}
\label{fig:three_horizontal}
\end{figure}

\subsubsection{d)}
\begin{figure}[htbp]
    \centering
    \includegraphics[width=0.5\linewidth]{imgs/Figure_1.png}
    \caption{Bézier curve with control points $P_0=(-1,0)$, $P_1=(0,1)$, $P_2=(2,0)$}
\end{figure}

\subsubsection{e)}

\subsubsection{f)}

\subsubsection{g)}

\begin{align*}
\mat{P'} = \mat{Q'} \mat{P} &=
\mtx{
     1& 0& 0\\
    1-u& u& 0\\
    (1-u)^2& 2u(1-u)& u^2
}
\mat{P}
&=
\mtx{
     1& 0& 0\\
    0.6& 0.4& 0\\
    0.6^2& 0.8(0.6)& 0.4^2
}
\mat{P}\\
&=
\mtx{
     1& 0& 0\\
    0.6& 0.4& 0\\
    0.36& 0.48& 0.16
}
\mtx{
    -1&0\\
    0&1\\
    2&0
} &=
\mtx{
    -1&0\\
    -0.6&0.4\\
    -0.04&0.48
}
\\
\mat{P''} = \mat{Q''} \mat{P} &=
\mtx{
    (1-u)^2& 2u(1-u)& u^2\\
    0& 1-u& u\\
    0& 0& 1
}
\mat{P}
&=
\mtx{
    0.6^2& 0.8(0.6)& 0.4^2\\
    0& 0.6& 0.4\\
    0& 0& 1
}
\mat{P}\\
&=
\mtx{
    0.36& 0.48& 0.16\\
    0& 0.6& 0.4\\
    0& 0& 1
}
\mtx{
    -1&0\\
    0&1\\
    2&0
} &=
\mtx{
    -0.04&0.48\\
    0.8&0.6\\
    2&0
}
\end{align*}

\subsubsection{h)}
A degree 4 subdivision follows directly from g) by concatenating the two sub-curves:
$$
\mtx{
    -1&0\\
    -0.6&0.4\\
    -0.04&0.48\\
    0.8&0.6\\
    2&0
}
$$
\end{document}